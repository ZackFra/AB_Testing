\PassOptionsToPackage{unicode=true}{hyperref} % options for packages loaded elsewhere
\PassOptionsToPackage{hyphens}{url}
%
\documentclass[]{article}
\usepackage{lmodern}
\usepackage{amssymb,amsmath}
\usepackage{ifxetex,ifluatex}
\usepackage{fixltx2e} % provides \textsubscript
\ifnum 0\ifxetex 1\fi\ifluatex 1\fi=0 % if pdftex
  \usepackage[T1]{fontenc}
  \usepackage[utf8]{inputenc}
  \usepackage{textcomp} % provides euro and other symbols
\else % if luatex or xelatex
  \usepackage{unicode-math}
  \defaultfontfeatures{Ligatures=TeX,Scale=MatchLowercase}
\fi
% use upquote if available, for straight quotes in verbatim environments
\IfFileExists{upquote.sty}{\usepackage{upquote}}{}
% use microtype if available
\IfFileExists{microtype.sty}{%
\usepackage[]{microtype}
\UseMicrotypeSet[protrusion]{basicmath} % disable protrusion for tt fonts
}{}
\IfFileExists{parskip.sty}{%
\usepackage{parskip}
}{% else
\setlength{\parindent}{0pt}
\setlength{\parskip}{6pt plus 2pt minus 1pt}
}
\usepackage{hyperref}
\hypersetup{
            pdftitle={Proofs},
            pdfborder={0 0 0},
            breaklinks=true}
\urlstyle{same}  % don't use monospace font for urls
\usepackage[margin=1in]{geometry}
\usepackage{graphicx,grffile}
\makeatletter
\def\maxwidth{\ifdim\Gin@nat@width>\linewidth\linewidth\else\Gin@nat@width\fi}
\def\maxheight{\ifdim\Gin@nat@height>\textheight\textheight\else\Gin@nat@height\fi}
\makeatother
% Scale images if necessary, so that they will not overflow the page
% margins by default, and it is still possible to overwrite the defaults
% using explicit options in \includegraphics[width, height, ...]{}
\setkeys{Gin}{width=\maxwidth,height=\maxheight,keepaspectratio}
\setlength{\emergencystretch}{3em}  % prevent overfull lines
\providecommand{\tightlist}{%
  \setlength{\itemsep}{0pt}\setlength{\parskip}{0pt}}
\setcounter{secnumdepth}{0}
% Redefines (sub)paragraphs to behave more like sections
\ifx\paragraph\undefined\else
\let\oldparagraph\paragraph
\renewcommand{\paragraph}[1]{\oldparagraph{#1}\mbox{}}
\fi
\ifx\subparagraph\undefined\else
\let\oldsubparagraph\subparagraph
\renewcommand{\subparagraph}[1]{\oldsubparagraph{#1}\mbox{}}
\fi

% set default figure placement to htbp
\makeatletter
\def\fps@figure{htbp}
\makeatother


\title{Proofs}
\author{}
\date{\vspace{-2.5em}}

\begin{document}
\maketitle

Consider data for variable x = \(x_1\), \(x_2\), . . . , \(x_n\). We use
x to denote the sample mean of x, and \(s_x\) is the sample standard
deviation of x.

\hypertarget{part-i}{%
\section{Part I}\label{part-i}}

For each of the following three transformations derive (a) the sample
mean z, and (b) the sample standard deviation \(s_z\).

\begin{enumerate}
\def\labelenumi{\arabic{enumi}.}
\tightlist
\item
  Centering
\end{enumerate}

\(z_i\) = (\(x_i\) − x)

\$\$

\def\zbar{\overline{z}}
\def\sumn{\sum_{i=1}^{n}}
\def\xbar{\overline{x}}

\overline{z}= \frac{1}{n}\sum(z\_i)\textbackslash{} =
\frac{1}{n}\sum(x\_i-\overline{x})\textbackslash{} =
\frac{1}{n}\sum(x\_i)-\frac{1}{n}\sum(\overline{x})\textbackslash{}
=\overline{x}-\overline{x}\textbackslash{} =0\textbackslash{}

S\_z\textsuperscript{2=\frac{1}{n}\sum(z\_i-\overline{z})\textbackslash{}
=\frac{1}{n}\sum(z\_i-0)}2\textbackslash{}
=\frac{1}{n}\sum(z\_i)\textsuperscript{2\textbackslash{}
=\frac{1}{n}\sum(x\_i-\overline{x})}2\textbackslash{}
=\sigma\^{}2(x)\textbackslash{}

S\_z=\sqrt(S\_z\textsuperscript{2)\textbackslash{}
=\sqrt(\sigma}2(x))\textbackslash{} =\sigma(x) \[
2. Scaling
\]

z\_i = \frac{x_i}{S_x}\textbackslash{} \textbackslash{}
\overline{z}=\frac{1}{n}\sum(z\_i)\textbackslash{}
=\frac{1}{n}\sum(\frac{x_i}{S_x})\textbackslash{}
=\frac{1}{n}\frac{1}{S_x}\sum(x\_i)\textbackslash{}
=\frac{\overline{x}}{S_x}\textbackslash{}

S\_z\textsuperscript{2=\frac{1}{n}\sum(z\_i-\overline{z})}2\textbackslash{}
=\frac{1}{n}\sum(\frac{x_i}{S_x}-\frac{\overline{x}}{S_x})\textsuperscript{2\textbackslash{}
=\frac{1}{n}\sum(\frac{x_i^2}{S_x^2}-\frac{2x_i\overline{x}}{S_x^2}+\frac{\overline{x}^2}{s_x^2})\textbackslash{}
=\frac{1}{S_x^2}\frac{1}{n}\sum(x\_i}2-2x\_i\overline{x}+\overline{x}\textsuperscript{2)\textbackslash{}
=\frac{1}{S_x^2}\frac{1}{n}\sum(x\_i-\overline{x})}2\textbackslash{}
=\frac{1}{S_x^2}S\_x\^{}2\textbackslash{} =1\textbackslash{}

S\_z=\sqrt(S\_z\^{}2)=1 \[
3. Centering and scaling (standardizing)
\] z\_i = \frac{(x_i − x)}{S_x} \[
\]

\$\$

\end{document}
